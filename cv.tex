%%%%%%%%%%%%%%%%%
% This is an sample CV template created using altacv.cls
% (v1.6.4, 13 Nov 2021) written by LianTze Lim (liantze@gmail.com). Now compiles with pdfLaTeX, XeLaTeX and LuaLaTeX.
%
%% It may be distributed and/or modified under the
%% conditions of the LaTeX Project Public License, either version 1.3
%% of this license or (at your option) any later version.
%% The latest version of this license is in
%%    http://www.latex-project.org/lppl.txt
%% and version 1.3 or later is part of all distributions of LaTeX
%% version 2003/12/01 or later.
%%%%%%%%%%%%%%%%

%% Use the "normalphoto" option if you want a normal photo instead of cropped to a circle
% \documentclass[10pt,a4paper,normalphoto]{altacv}

\documentclass[10pt,a4paper,ragged2e,withhyper]{altacv}
%% AltaCV uses the fontawesome5 and packages.
%% See http://texdoc.net/pkg/fontawesome5 for full list of symbols.

% Change the page layout if you need to
\geometry{left=1.25cm,right=1.25cm,top=1.5cm,bottom=1.5cm,columnsep=1.2cm}

% The paracol package lets you typeset columns of text in parallel
\usepackage{paracol}

% Change the font if you want to, depending on whether
% you're using pdflatex or xelatex/lualatex
\ifxetexorluatex
  % If using xelatex or lualatex:
  \setmainfont{Roboto Slab}
  \setsansfont{Lato}
  \renewcommand{\familydefault}{\sfdefault}
\else
  % If using pdflatex:
  \usepackage[rm]{roboto}
  \usepackage[defaultsans]{lato}
  % \usepackage{sourcesanspro}
  \renewcommand{\familydefault}{\sfdefault}
\fi

% Change the colours if you want to
\definecolor{SlateGrey}{HTML}{2E2E2E}
\definecolor{LightGrey}{HTML}{666666}
\definecolor{DarkPastelRed}{HTML}{450808}
\definecolor{PastelRed}{HTML}{8F0D0D}
\definecolor{GoldenEarth}{HTML}{E7D192}
\colorlet{name}{black}
\colorlet{tagline}{PastelRed}
\colorlet{heading}{DarkPastelRed}
\colorlet{headingrule}{GoldenEarth}
\colorlet{subheading}{PastelRed}
\colorlet{accent}{PastelRed}
\colorlet{emphasis}{SlateGrey}
\colorlet{body}{LightGrey}

% Change some fonts, if necessary
\renewcommand{\namefont}{\Huge\rmfamily\bfseries}
\renewcommand{\personalinfofont}{\footnotesize}
\renewcommand{\cvsectionfont}{\LARGE\rmfamily\bfseries}
\renewcommand{\cvsubsectionfont}{\large\bfseries}


% Change the bullets for itemize and rating marker
% for \cvskill if you want to
\renewcommand{\itemmarker}{{\small\textbullet}}
\renewcommand{\ratingmarker}{\faCircle}

%% Use (and optionally edit if necessary) this .cfg if you
%% want to use an author-year reference style like APA(6)
%% for your publication list
\input{pubs-authoryear.cfg}

%% Use (and optionally edit if necessary) this .cfg if you
%% want an originally numerical reference style like IEEE
%% for your publication list
% \input{pubs-num.cfg}

%% sample.bib contains your publications
\addbibresource{sample.bib}

\begin{document}
\name{Dr. Martin Swientek}
\tagline{I lead innovative cloud projects to success}
%% You can add multiple photos on the left or right
\photoR{2.8cm}{profilbild_neu}
% \photoL{2.5cm}{Yacht_High,Suitcase_High}

\personalinfo{%
  % Not all of these are required!
  \email{martin@swientek.org}
  \phone{+49 162 2906385}
  \mailaddress{Londoner Str. 2, 60327 Frankfurt, Germany}
  \location{Frankfurt / Main, Germany}
  \linkedin{dr-martin-swientek}
  \github{mswiente}
  %% You can add your own arbitrary detail with
  %% \printinfo{symbol}{detail}[optional hyperlink prefix]
  % \printinfo{\faPaw}{Hey ho!}[https://example.com/]
  %% Or you can declare your own field with
  %% \NewInfoFiled{fieldname}{symbol}[optional hyperlink prefix] and use it:
  % \NewInfoField{gitlab}{\faGitlab}[https://gitlab.com/]
  % \gitlab{your_id}
  %%
  %% For services and platforms like Mastodon where there isn't a
  %% straightforward relation between the user ID/nickname and the hyperlink,
  %% you can use \printinfo directly e.g.
  % \printinfo{\faMastodon}{@username@instace}[https://instance.url/@username]
  %% But if you absolutely want to create new dedicated info fields for
  %% such platforms, then use \NewInfoField* with a star:
  % \NewInfoField*{mastodon}{\faMastodon}
  %% then you can use \mastodon, with TWO arguments where the 2nd argument is
  %% the full hyperlink.
  % \mastodon{@username@instance}{https://instance.url/@username}
}

\makecvheader
%% Depending on your tastes, you may want to make fonts of itemize environments slightly smaller
% \AtBeginEnvironment{itemize}{\small}

%% Set the left/right column width ratio to 6:4.
\columnratio{0.6}

% Start a 2-column paracol. Both the left and right columns will automatically
% break across pages if things get too long.
\begin{paracol}{2}
\cvsection{Experience}

\cvevent{Managing Delivery Architect}{Capgemini}{Febr. 2008 -- Ongoing}{Frankfurt, Germany}
\begin{itemize}
\item Member of the Cloud Center of Excellence helping clients to adopt Cloud technologies
\item Working together with the sales organization on bids as lead architect responsible for the solution
\item Supporting an automotive company to build a sovereign data space aligned with the principles of Gaia-X
\item Built and operated a microservice-based Cloud Integration Platform as a Managed Service (iPaaS) for an energy provider generating more than 2M EUR revenue per year with an onsite/offshore team of up to 14 FTE based on a DevOps model. The solution has been used as a success story and a customer reference for the Gartner Magic Quadrant.
\item Built a seal of approval for trusted cloud services to foster the cloud adoption of SMEs initiated by the Federal Ministry of Economic Affairs and Energy, defined a criteria catalogue for trusted cloud services and conducted reviews to award the seal.
\end{itemize}

\divider

\cvevent{Systems Engineer}{T-Systems}{2003 -- 2008}{Darmstadt, Germany}
\begin{itemize}
\item Designed and developed a system for bill formatting and processing for the mass market for a large telecommunications provider that processed more than 30 million bills per month
\end{itemize}

\cvsection{Certification}

\begin{itemize}
\item Cloud Security Professional (CCSP)
\item Professional Scrum Master (PSM I)
\end{itemize}

\medskip

\cvsection{A Day of My Life}

% Adapted from @Jake's answer from http://tex.stackexchange.com/a/82729/226
% \wheelchart{outer radius}{inner radius}{
% comma-separated list of value/text width/color/detail}
\wheelchart{1.5cm}{0.5cm}{%
  25/8em/accent!70/{Sleeping \& Dreaming},
  10/8em/accent!10/Business development,
  15/8em/accent!20/Client work,
  10/10em/accent!30/Olympic Weightlifting \& Crossfit,
  10/10em/accent!40/Learning something new,
  20/6em/accent!50/Spending time with my family,
  10/6em/accent!60/Feeding the cats
}

%% Switch to the right column. This will now automatically move to the second
%% page if the content is too long.
\switchcolumn

\cvsection{My Life Philosophy}

\begin{quote}
``Light weight, Baby!'' - Ronnie Coleman
\end{quote}

\cvsection{Most Proud of}

\cvachievement{\faTrophy}{Finishing my PhD}{with a fulltime job}

\divider

\cvachievement{\faHeartbeat}{My little son}{and my family}

\cvsection{Strengths}

\cvtag{Strong technical understanding}\\
\cvtag{Software Engineering background}\\
\cvtag{Effective communication}
\cvtag{Persistence}
\cvtag{Teamplayer}
\cvtag{Hands-on mentality}\\
\cvtag{Building trust}
\cvtag{Client intimacy}

\divider\smallskip

\cvtag{Amazon Webservices}
\cvtag{Cloud Architecture}\\
\cvtag{Cloud Integration Platforms}
\cvtag{Scrum}
\cvtag{SAFe}
\cvtag{Restful APIs}

\cvsection{Languages}

\cvskill{German}{5}
\divider

\cvskill{English}{4}

%% Yeah I didn't spend too much time making all the
%% spacing consistent... sorry. Use \smallskip, \medskip,
%% \bigskip, \vspace etc to make adjustments.
\medskip

\cvsection{Education}

\cvevent{Ph.D.}{Plymouth University\\School of Engineering, Computing and Mathematics }{2008 -- 2015}{}
Thesis title: High-Performance Near-Time Processing of Bulk Data

\divider

\cvevent{M.Sc.\ in Computer Science}{Hochschule Darmstadt}{2005 -- 2008}{}

\divider

\cvevent{B.Sc.\ in Computer Science}{Hochschule Darmstadt}{2000 -- 2005}{}

% \divider

\end{paracol}


\end{document}
